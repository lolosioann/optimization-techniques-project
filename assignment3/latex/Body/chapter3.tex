\chapter{Η μέθοδος Μέγιστης Καθόδου με Προβολή}

\section{Βασικό Θεωρητικό Πλαίσιο}
Η μέθοδος Μέγιστης Καθόδου με Προβολή (Maximum Descent with Projection) αποτελεί τεχνική βελτιστοποίησης που επιλύει προβλήματα περιορισμένης ελαχιστοποίησης. Αποτελεί τροποποίηση της απλής μεθόδου Μέγιστης Καθόδου, ώστε να επιτρέπει την εφαρμογή της σε προβλήματα με περιορισμούς. Η σύγκλιση της επηρεάζεται από παράγοντες όπως η επιλογή του βήματος $\gamma$ και η κυρτότητα της συνάρτησης.

\section{Μεθοδολογία}
Η παρούσα μέθοδος "βρίσκει" τα νέα εφικτά σημεία με την εξής σχέση \cite{texnbelt}:
\begin{equation}
    x_{k+1} = x_k + \gamma_k \cdot (\Bar{x} - x_k)    
\end{equation}
όπου
\begin{equation}
    \Bar{x} = Pr_x\{x_k- s_k\nabla f(x_k)\}    
\end{equation}
Το $s_k$ επιλέγεται από εμάς, ενώ ο τελεστής $Pr_x(.)$ ουσιαστικά "προβάλλει" οποιοδήποτε σημείο εκτός των περιορισμών στο κοντινότερο του εντός των περιορισμών. Προφανώς τα σημεία που βρίσκονται ήδη μέσα στους περιορισμούς δεν επηρεάζονται από τον τελεστή.\par

Με τον παραπάνω τρόπο εγγυόμαστε πως η αναζήτηση ελαχίστου παραμένει πάντοτε εντός της περιοχής που επιβάλλουν οι περιορισμοί.