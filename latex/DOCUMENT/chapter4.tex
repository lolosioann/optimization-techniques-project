\chapter{Σύγκριση των Μεθόδων - Συμπεράσματα}
\label{ch:chapter4}

\begin{itemize}
    \item Είναι εμφανές πως όσο μειώνεται το $l$ οι αλγόριθμοι απαιτούν περισσότερες επαναλήψεις μέχρι τον τερματισμό τους, ωστόσο επιτυγχάνεται σαφώς μεγαλύτερη ακρίβεια προσέγγισης του ελαχίστου της συνάρτησης.
    \item Αντίστροφα, στην μέθοδο της Διχοτόμου χωρίς χρήση παραγώγων, παρατηρούμε ότι όσο μεγαλώνουμε το $e$ ο αλγόριθμος απαιτεί περισσότερες επαναλήψεις για να υπολογίσει το ελάχιστο, καθώς το διάστημα $[\alpha_k , \beta_k]$ πλησιάζει πιο αργά στην ακρίβεια $l$.
    \item Σύμφωνα με τα γραφήματα η μέθοδος διχοτόμου (με και χωρίς χρήση παραγώγων) απαιτεί λιγότερες επαναλήψεις από τις άλλες μεθόδους, ωστόσο η μέθοδος χωρίς παραγώγους το επιτυγχάνει χρησιμοποιώντας περισσότερους υπολογισμούς της $f$.
    \item Όσον αφορά τις συναρτήσεις δεν παρατηρήθηκε κάποια σημαντική διαφορά ανάμεσα τους στις υπολογιστικές απαιτήσεις μέχρι την εύρεση του ελαχίστου τους.
\end{itemize}


